	\begin{tabular}{ | c   p{18cm} |}
		\hline
		\cellcolor{black}\rotcell{\large\textbf{\textcolor{white}{Magnetostatik}}}  &
		\setlength{\extrarowheight}{10pt}	
		
		\begin{tabular}{L{5cm} L{6.5cm} L{5.3cm}}
			&&\\[-20pt]
			\rowcolor[rgb]{1,1,1}
			
			
			\textbf{Magnetische Flussdichte} \qquad\qquad\qquad Gesetzt von Biot-Savart Ladungsdichte&$\displaystyle \mathbf{B}=\frac{\mu_{0}}{4 \pi} \int_{V} \frac{(\mathbf{J} d v) \times \hat{\mathbf{R}}}{R^{2}}=\frac{\mu_{0}}{4 \pi} \int_{L} \frac{(I d \mathbf{l}) \times \hat{\mathbf{R}}}{R^{2}} $&$\displaystyle [B]=\mathrm{T}=\frac{\mathbf{W b}}{\mathrm{m}^{2}}$\\[5pt]
			
			
			\rowcolor[rgb]{0.91,0.91,0.91}
			\textbf{Magnetische Feldstärke} \qquad\qquad Magnetische Hilfsgrösse&$\displaystyle \mathbf{H}=\frac{\mathbf{B}}{\mu_{0}}-\mathbf{M} $&$\displaystyle \left[H\right]=\frac{\mathrm{A}}{\mathrm{m}}$\\[5pt]
	
			\rowcolor[rgb]{1,1,1}
			\textbf{Magn. Materialgleichung} \qquad\qquad\qquad magn. Permeabilität$\mu$&
			$\displaystyle \mathbf{B}=\mu \mathbf{H}=\mu_{0} \mu_{r} \mathbf{H}=\mu_{0}(\mathbf{H}+\mathbf{M}) $&$\displaystyle \left(\mu_{r}=1+\chi_{m}\right)$\\[5pt]
			
				\rowcolor[rgb]{0.91,0.91,0.91}
			\textbf{ Magnet. Vektorpotential} &$\displaystyle \mathbf{A}=\frac{\mu_{0}}{4 \pi} \int_{V} \frac{\mathbf{J} d v}{R}=\frac{\mu_{0}}{4 \pi} \int_{L} \frac{I d \mathbf{l}}{R}$&$\left[ A\right ]= Wb/m$ (nicht Fläche A)\\[5pt]
			
			\rowcolor[rgb]{1,1,1}
			\textbf{Magnetische Spannung} \qquad\qquad\qquad und Umlaufsspannung$\mu$&
			$\displaystyle V_{m}=\int_{L} \mathbf{H} \cdot d \mathbf{l}, \quad \stackrel{\circ}{V}_{m}=\oint_{C} \mathbf{H} \cdot d \mathbf{l}$
			& $\displaystyle \left[V_m\right]=A$\\[5pt]
			
			
			\rowcolor[rgb]{0.91,0.91,0.91}
			\textbf{Magnet. Dipolmoment}\qquad\qquad (einer Stromschleife) &$\displaystyle \mathbf{m}=\hat{\mathbf{n}} I A \quad(\hat{\mathbf{n}} \perp A)  \Rightarrow \quad \mathbf{T}=\mathbf{m} \times \mathbf{B} $&$[m]=\mathrm{A} \mathrm{m}^{2}$\\[5pt]
			
			
			\rowcolor[rgb]{1,1,1}
			\textbf{Magnetisierung } \qquad\qquad\qquad (im linearen Bereich)&
			$\displaystyle \mathbf{M}=\chi_{m} \mathbf{H}=\frac{d}{d v}\left[\sum_{n} \mathbf{m}_{n}\right]_{\text {in } d v}=N_{m} \mathbf{m}$
			& $\displaystyle [m]=\frac{\mathrm{A}}{\mathrm{m}} $\\[5pt]
			
			
			
			\rowcolor[rgb]{0.91,0.91,0.91}
			\textbf{Magnetischer Fluss}\qquad\qquad und magn. Widerstand $R_m$ &$\displaystyle \Phi=\int_{A} \mathbf{B} \cdot d \mathbf{s}=\oint_{C=\partial A} \mathbf{A} \cdot d \mathbf{l}=\frac{V_{m}}{R_{m \mid}}=V_{m} \Lambda$&$[\Phi]=Wb=Tm^2$\\[5pt]
					
					
			\rowcolor[rgb]{1,1,1}
			\textbf{Induktivität } \qquad\qquad\qquad Selbst- \& Gegeninduktivität&
			$\displaystyle L=\frac{\Phi_{1}}{I_{1}}=L_{a}+L_{i}, \quad M=\frac{\Phi_{2}}{I_{1}}=k \sqrt{L_{1} L_{2}}$& $\displaystyle [L]=[M]=H$ \qquad (nicht Magnet.M!) \\[5pt]		
				
			\rowcolor[rgb]{0.91,0.91,0.91}
			\textbf{Gespeicherte Energie}\qquad\qquad einer Induktivität &$\displaystyle W_{m}=\frac{1}{2} L I^{2}=\frac{1}{2} \frac{\Phi^{2}}{L}$&$\displaystyle [W_m]=J=Ws$\\[5pt]	
			
			
			
			\rowcolor[rgb]{1,1,1}
			\textbf{Energiedichte } \qquad\qquad\qquad des magnetischen Felds&
			$\displaystyle w_{m}=\frac{d W_{m}}{d v}=\frac{1}{2} \mathbf{B} \cdot \mathbf{H}=\frac{1}{2} \mu H^{2}=\frac{B^{2}}{2 \mu}$& $\displaystyle [w_m]=\frac{J}{m^3}$ \\[5pt]	
					
					
				
			\rowcolor[rgb]{0.91,0.91,0.91}
			\textbf{Magnetkraft}\qquad\qquad Maxwellsche Zugkraftformel &$\displaystyle \mathbf{F}=\frac{d W_{m}}{d l} \hat{\mathbf{l}}=\frac{B^{2} A}{2 \mu_{0}} \hat{\mathbf{n}} \quad(\hat{\mathbf{n}} \perp A)$&$\displaystyle [F]=N$\\[5pt]
			
			\rowcolor[rgb]{1,1,1}
			\textbf{Ampèresche Gesetze} \qquad\qquad\qquad Durchflutungssatz $\Theta =\stackrel{\circ}{V}_{m}$&\multicolumn{2}{l}{
			$\displaystyle \oint_{C=\partial A} \mathbf{B} \cdot d \mathbf{l}=\mu_{0} \int_{A} \mathbf{J} \cdot d \mathbf{s}, \quad \oint_{C=\partial A} \mathbf{H} \cdot d \mathbf{l}=\int_{A} \mathbf{J}_{\mathbf{f r e i}} \cdot d \mathbf{s}=\Theta$} \\[5pt]	
		
			\rowcolor[rgb]{0.91,0.91,0.91}
			\textbf{Gausssches Gesetz}\qquad\qquad Flusskontinuität &\multicolumn{2}{l}{
				$\displaystyle \oint_{\text{Hülle}} \mathbf{B} \cdot d \mathbf{s}=0$ \qquad $\Leftrightarrow$\qquad$\displaystyle \sum_{n} \Phi_n =0$ }
			\\[5pt]
					
		\end{tabular}\\
		\hline
	\end{tabular}