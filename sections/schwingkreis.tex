	\begin{tabular}{ | c   p{18cm} |}
		\hline
		\cellcolor{black}\rotcell{\large\textbf{\textcolor{white}{Schwingkreise}}}  &
		\setlength{\extrarowheight}{10pt}	
		
		\begin{tabular}{L{5cm} L{11.8cm}}
			&\\[-20pt]
			
			\rowcolor[rgb]{1,1,1}
			\textbf{Kreisgüte}  \qquad\qquad\qquad Serie-SK: $Q_S$, Parallel-SK:$Q_P$
			&$\displaystyle Q=\frac{1}{B_{\mathrm{rel}}}=\frac{\omega_{m} W}{P} \Rightarrow Q_{S}=\frac{1}{R} \sqrt{\frac{L}{C}}, Q_{P}=\frac{1}{G} \sqrt{\frac{C}{L}}$ \\[5pt]
			
			\rowcolor[rgb]{0.91,0.91,0.91}
			\textbf{Resonanzfrequenz}  \qquad\qquad Eigenfrequenz $\omega_0 = \omega_{r \text{verlustlos}}$
			&$\displaystyle \omega_{r}: \operatorname{Im} Z\left(\omega_{r}\right)=0 \quad$ bzw. $\quad \operatorname{Im} Y\left(\omega_{r}\right)=0$ \\[5pt]
	
			\rowcolor[rgb]{1,1,1}
			\textbf{Extremalfrequenz}  \qquad\qquad Bandbreite$B = \omega_m \pm \omega_{3dB}$
			&$\displaystyle \omega_{m}=\arg \max _{\omega}|Z(\omega)| \quad$ bzw. $\quad \omega_{m}=\arg \max _{\omega}|Y(\omega)|$ \\[5pt]
		
		
		
			\rowcolor[rgb]{0.91,0.91,0.91}
			\textbf{Dämpfungsgrad}  \qquad\qquad Zeitkonstante $\tau$
			&$\displaystyle  \zeta=\frac{1}{2 Q}=\frac{1}{\omega_{0} \tau} \qquad\qquad\qquad [\zeta]=1$ \\[5pt]		
			
			\rowcolor[rgb]{1,1,1}
			\textbf{Natürliche Frequenz}  \qquad\qquad Gedämpfte Eigenfrequenz
			&$\displaystyle \omega_{n}=\omega_{0} \sqrt{1-\zeta^{2}} \qquad\qquad\qquad\left[\omega_{n}\right]=\mathrm{rad} / \mathrm{s}$ \\[5pt]
			
			\rowcolor[rgb]{0.91,0.91,0.91}
			\textbf{Natürliche Schwingung}  \qquad\qquad Freies Ausschwingen 
			&$\displaystyle  a(t)=a_{0} \mathrm{e}^{-t / \tau} \sin \left(\omega_{n} t+\phi\right) \quad a(t)=u(t)$ bzw. $i(t)$ \\[5pt]
			
			
				\rowcolor[rgb]{1,1,1}
			\textbf{Verstimmung}
				 \qquad\qquad normierte(r) Frequenz(gang) 
			&$\displaystyle \nu=\frac{\omega}{\omega_{m}}-\frac{\omega_{m}}{\omega} \qquad\qquad \Omega=\nu Q \qquad\qquad \frac{Z}{R}=1+j \Omega$ \\[5pt]
			
			
			
		\end{tabular}\\
		\hline
	\end{tabular}