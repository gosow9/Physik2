			
			
\begin{tabular}{ | c   p{18cm} |}
				\hline
				%\rowcolor{blue!30}
				\cellcolor{black}\rotcell{\large\textbf{\textcolor{white}{Maxwell-Gleichungen}}}  &
				\setlength{\extrarowheight}{5pt}	
				
				\begin{tabular}{L{4cm} R{4.2cm} C{0.5cm} L{6.3cm} L{0.9cm}}
					&&&&\\[-10pt]
					
					\textbf{Physikalisches gaußsches Gesetz}&
					$\displaystyle \operatorname{div} \vec{D}=\vec{\nabla} \cdot \vec{D}=\rho$&$\displaystyle \Longleftrightarrow$& $\displaystyle \oint_{\partial V} \vec{D} \cdot \mathrm{d} \vec{A}=\iiint_{V} \rho \mathrm{d} V=Q(V)$ & Gauss\\[10pt]
					
				\multicolumn{2}{l}{\parbox{8.2cm}{Das $\displaystyle \vec {D}$-Feld ist ein Quellenfeld. Die Ladung (Ladungsdichte ${\displaystyle \rho }$) ist Quelle des elektrischen Feldes.}}&&\multicolumn{2}{l}{\parbox{6.5cm}{Der (elektrische) Fluss durch die geschlossene Oberfläche $\displaystyle \partial V$ eines Volumens $V$ ist direkt proportional zu der elektrischen Ladung in seinem Inneren.}}\\	[10pt]
					
					
				\rowcolor[rgb]{0.91,0.91,0.91}
				\textbf{Quellenfreiheit des B-Feldes}&
				$\displaystyle\operatorname{div} \vec{B}=\vec{\nabla} \cdot \vec{B}=0 $&$\Longleftrightarrow$&$\displaystyle \oint_{\partial V} \vec{B} \cdot \mathrm{d} \vec{A}=0$ & Gauss\\[10pt]
				
				\multicolumn{2}{l}{\parbox{8.2cm}{Das $\vec {B}$-Feld ist quellenfrei. Es gibt keine magnetischen Monopole.}}&&
				\multicolumn{2}{l}{\parbox{6.5cm}{Der magnetische Fluss durch die geschlossene Oberfläche eines Volumens ist gleich der magnetischen Ladung in seinem Inneren, nämlich Null, da es keine magnetischen Monopole gibt.}}\\	[10pt]
				
				
					
					
					
				\rowcolor[rgb]{1,1,1}	
				\textbf{Induktionsgesetz}&
				$\displaystyle\operatorname{rot} \vec{E}=\vec{\nabla} \times \vec{E}=-\frac{\partial \vec{B}}{\partial t} $&$\Longleftrightarrow$&$\displaystyle \oint_{\partial A} \vec{E} \cdot \mathrm{d} \vec{s}=-\iint_{A} \frac{\partial \vec{B}}{\partial t} \cdot \mathrm{d} \vec{A}$ & Stokes\\[10pt]
				
				
				\multicolumn{2}{l}{\parbox{8.2cm}{Jede Änderung des $\vec {B}$-Feldes führt zu einem elektrischen Gegenfeld. Die Wirbel des elektrischen Feldes sind von der zeitlichen Änderung der magnetischen Flussdichte abhängig.}}&&
				\multicolumn{2}{l}{\parbox{6.5cm}{Die (elektrische) Zirkulation über der Randkurve $\partial A$ einer Fläche $A$ ist gleich der negativen zeitlichen Änderung des magnetischen Flusses durch die Fläche.}}\\[10pt]	
				
					
				\rowcolor[rgb]{0.91,0.91,0.91}
				\textbf{Durchflutungsgesetz}&
				$\displaystyle \operatorname{rot} \vec{H}=\vec{\nabla} \times \vec{H}=\vec{j}_{1}+\frac{\partial \vec{D}}{\partial t}$&$ \Longleftrightarrow$&$\displaystyle \oint_{\partial A} \vec{H} \cdot \mathrm{d} \vec{s}=\iint_{A} \vec{j}_{1} \cdot \mathrm{d} \vec{A}+\iint_{A} \frac{\partial \vec{D}}{\partial t} \cdot \mathrm{d} \vec{A}$ & Stokes\\[10pt]
				
				\multicolumn{2}{l}{\parbox{8.2cm}{Die Wirbel des Magnetfeldes hängen von der Leitungsstromdichte $\vec {j}_{\mathrm {l}}$ und von der elektrischen Flussdichte $\vec {D}$ ab. Die zeitliche Änderung von $\vec {D}$ wird auch als Verschiebungsstromdichte $\vec {j}_{\mathrm {v}}$ bezeichnet und ergibt als Summe mit der Leitungsstromdichte die totale Stromdichte $\vec {j}_{\text{total}}=\vec {j}_{\mathrm {l} }+\vec {j}_{\mathrm {v}}$}}&&
				\multicolumn{2}{l}{\parbox{6.5cm}{Die magnetische Zirkulation über der Randkurve $\partial A$ einer Fläche $A$ ist gleich der Summe aus dem Leitungsstrom und der zeitlichen Änderung des elektrischen Flusses durch die Fläche.}}\\[10pt]	
				
					
				\end{tabular}\\
				\hline
			
		\end{tabular}
