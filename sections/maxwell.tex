
		\begin{tabular}{ | c   p{18cm} |}
			\hline
			%\rowcolor{blue!30}
			\cellcolor{black}\rotcell{\large\textbf{\textcolor{white}{Maxwell-Gleichungen}}}  &
			\setlength{\extrarowheight}{5pt}	
					
			\begin{tabular}{L{5cm} L{6.5cm} L{5.3cm}}
				
			\textbf{Faradaysches Gesetz}("Totalinduktion")&
			$\displaystyle \oint_{C=\partial A} \mathbf{E} \cdot d \mathbf{l}=-\frac{d}{d t} \int_{A} \mathbf{B} \cdot d \mathbf{s}$
			&
			$u_{i}(t)=-\frac{d \Phi}{d t}(t)$ \\[5pt]
			
			&$\nabla \times \boldsymbol{B}=\mu_{0} \boldsymbol{J}+\mu_{0} \varepsilon_{0} \frac{\partial \boldsymbol{E}}{\partial t}$
			&\\[5pt]
				
				
			\rowcolor[rgb]{0.91,0.91,0.91}
			\textbf{Ampèresches Gesetz}\qquad(vollst. Durchflutungssatz) &
			$\displaystyle \oint_{C=\theta A} \mathbf{H} \cdot d \mathbf{l}=\int_{A}\left(\mathbf{J}_{\text {frei }}+\frac{d \mathbf{D}}{d t}\right) \cdot d \mathbf{s}$&
			$\stackrel{\circ}{V}_{m}(t)=\Theta(t)+\frac{d \Psi}{d t}(t)$\\[10pt]
				
				
			\rowcolor[rgb]{1,1,1}\textbf{Gausssches Gesetz}\qquad\qquad(des elektrischen Felds)&
			$\displaystyle \oint_{A=\partial V} \mathbf{D} \cdot d \mathbf{s}=\int_{V} \rho_{\mathrm{frei}} d v$&
			$\stackrel{\circ}{\Psi}(t)=Q_{\text {frei }}(t)$		\\[10pt]
				
			\rowcolor[rgb]{0.91,0.91,0.91}
			\textbf{Gausssches Gesetz}\qquad\qquad(des Magnetfelds) &
			$\displaystyle \oint_{A=\partial V} \mathbf{B} \cdot d \mathbf{s}=0$&
			$\stackrel{\circ}{\Phi}(t)=0$ \\[10pt]
			
				
			
	
			\end{tabular}\\
			\hline
		\end{tabular}
