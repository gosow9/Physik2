
		\begin{tabular}{ | c   p{18cm} |}
			\hline
			%\rowcolor{blue!30}
			\cellcolor{black}\rotcell{\large\textbf{\textcolor{white}{Elektrostatik}}}  &
			\setlength{\extrarowheight}{10pt}	
					
			\begin{tabular}{L{5cm} L{6.5cm} L{5.3cm}}
				
			\textbf{Elektrische Feldstärke}&
			$\displaystyle \vec{E}=\lim _{q \rightarrow 0} \frac{\vec{F}}{q} \qquad\qquad \vec{F}=q \vec{E}$
			&
			$\displaystyle[E]=\frac{\mathrm{V}}{\mathrm{m}}$ \\[5pt]
			für $n$ Ladungen&$\displaystyle \vec{E}(\vec{r}) = \frac{1}{4\pi\varepsilon_0} \sum_{i=1}^{n} \frac{q_i}{|\vec{r}-\vec{r}_i|^3}(\vec{r}-\vec{r}_i)$ &\\
			Ladungsverteilung&$\displaystyle	\vec{E}(\vec{r}) = \frac{1}{4\pi\varepsilon_0} \iiint_{\mathbb{R}^3} \frac{\rho(r')}{|\vec{r}-\vec{r}'|^3}(\vec{r}-\vec{r}') \diff x' \diff y' \diff z' $&(Ladungsdichte $\rho$)\\
			

			\textbf{Punktladung}&
			$\displaystyle \vec{E}(\vec{r}) = \frac{1}{4\pi\varepsilon_0} \frac{q\hat{r}}{r^2}$ & \\
			
			\textbf{Kugeloberfläche}&
			$\displaystyle \vec{E}(\vec{r})= \begin{cases}0 \qquad\qquad r<R \\ \displaystyle\frac{1}{4 \pi \varepsilon_{0}} \frac{q \hat{r}}{r^{2}} \qquad r>R\end{cases}$&
			
			\textbf{$\infty$-Draht}\qquad $\displaystyle \vec{E}(\vec{r})= \frac{\lambda}{2\pi\varepsilon_0r}$\\
		
			\textbf{$\infty$-Zylinder}&
			$\displaystyle \vec{E}(\vec{r})= \begin{cases}0 \qquad\qquad r<R \\ \displaystyle \frac{\lambda \hat{r}}{2 \pi \varepsilon_{0} r} \qquad r>R\end{cases}$&
			
	
			\textbf{$\infty$-Ebene}\qquad
			$\displaystyle \vec{E}(\vec{r})=\frac{\sigma}{2\varepsilon_0}$\\
			
			
			
							
			\rowcolor[rgb]{0.91,0.91,0.91}
			\textbf{Coulombsches Gesetz} &
			$\displaystyle \vec{F}_{21}=k \frac{q_{1} q_{2}}{r_{21}^{2}} \hat{r}_{21}$&
			 $\displaystyle k=\frac{1}{4 \pi \varepsilon_{0}}$\\[10pt]
				
				
			\rowcolor[rgb]{1,1,1}
			\textbf{Earnshaw-Theorem}&\multicolumn{2}{l}{\parbox{10cm}{Kein System stationärer Ladung ist in einem stabilen Gleichgewicht unter der alleinigen Wirkung elektrischer Kräfte}}
					\\[10pt]
				
			\rowcolor[rgb]{0.91,0.91,0.91}
			\textbf{Elektrische Flussdichte} 	& $D=\epsilon E = \epsilon_0 \epsilon_r E$\qquad$ (\epsilon_r = 1 + \chi_e)$   & $\displaystyle[D]= \frac{C}{m^2}$  \\[5pt]
			
			\rowcolor[rgb]{1,1,1}
			\textbf{Elektrischer Fluss}\qquad\qquad &
			$\displaystyle   \Psi_D = \int_{A}{ \vec{D}\cdot \diff\vec{s}}$\qquad $\displaystyle\Psi_E = \int_{A}{ \epsilon_0\vec{E}\cdot \diff\vec{s}}$&$[\Psi]= C (Coulomb) = As$ \\
			
			bei geschlossener Oberfläche& $\displaystyle \Psi_D=\oint_{\text{Hülle}} \vec{D} \cdot \mathrm{d} \vec{s}=Q_{umschlossen}$&
			$\Psi=CU=Q$\\[10pt]
			
			\rowcolor[rgb]{0.91,0.91,0.91}
			\textbf{Elektrischer Fluss} \qquad\qquad im Vakuum	&\multicolumn{2}{l}{
			$\displaystyle \diff \Phi = \vec{E}\cdot \diff\vec{a} \qquad \Rightarrow \qquad \Phi =\frac{\Psi}{\epsilon_0}= \int_{\del V} \vec{E}\cdot \diff\vec{a}$} \\[5pt]
			
			\rowcolor[rgb]{1,1,1}
			\textbf{Gaussches Gesetz}&
			$\displaystyle \Phi = \int_{\del V}\epsilon_0 \vec{E}\cdot \diff\vec{a} = \int_V \rho \diff V = \sum_V q=Q$&$\displaystyle \Psi_D=\oint_{\text{Hülle}} \vec{D} \cdot \mathrm{d} \vec{s}=Q_{frei}$ \\[5pt]
					
			\rowcolor[rgb]{0.91,0.91,0.91}
			\textbf{Energiedichte}	&
			$\displaystyle w = \frac{\diff W}{\diff V} = \frac{\varepsilon_0}{2}E^2$&\\[5pt]
			
			\rowcolor[rgb]{1,1,1}
			\textbf{Potentielle Energie}&
			$\displaystyle W =\int_V w \diff V = \int_V \frac{\varepsilon_0}{2}E^2 \diff V$&\\[5pt]
			
			\rowcolor[rgb]{0.91,0.91,0.91}
			\textbf{E-Feld konservativ}&
			$\displaystyle \oint \vec{E} \cdot \diff\vec{s} =0 \qquad \Leftrightarrow \qquad \vec{E} = -\nabla \phi$&\\[5pt]
			
			\rowcolor[rgb]{1,1,1}
			\textbf{Potentialdifferenz}&
			$\displaystyle U_{AB} =\phi(A) - \phi(B) = \int_{A}^{B} \vec{E}\cdot\diff\vec{s}$& $[U]=V$ \\[5pt]
			
			\rowcolor[rgb]{0.91,0.91,0.91}
			\textbf{Potential mehrere Ladungen}&
			$\displaystyle \phi(\vec{r}) = \frac{1}{4\pi\varepsilon_0} \iiint_{\mathbb{R}^3} \frac{\rho(\vec{r'})}{|\vec{r}-\vec{r'}|}\diff x' \diff y' \diff z'$&\\[5pt]
			
			
			
			\rowcolor[rgb]{1,1,1}
			\textbf{Potentielle Energie }\qquad \qquad einer Ladungsverteilung ($\rho$)&
			$\displaystyle	W = \frac{1}{2}\iiint_{\mathbb{R}^3} \phi(\vec{r})\rho(\vec{r})\diff x \diff y \diff z
			$&\\[5pt]
			
			
			\rowcolor[rgb]{0.91,0.91,0.91}
			& $\displaystyle U_{BA} = E\Delta z \quad$ & Plattenkondensator\\[-2pt]
			\textbf{Potentiale einfacher Ladungsverteilung}& $\displaystyle U_{BA} = \frac{1}{4\pi\varepsilon_0}\frac{q}{r}$ & Punktladung\\[-1pt]
			& $\displaystyle U_{BA} = \frac{1}{4\pi\varepsilon_0}\frac{q}{\sqrt{x^2 + R^2}}$ & Ringladung\\[-1pt]
			& $\displaystyle U_{BA} = \frac{\sigma}{2\varepsilon_0}\left( \sqrt{x^2+R^2}-x\right)$  & Scheibe\\[5pt]
			

	
			\end{tabular}\\
			\hline
		\end{tabular}
	
	
	
	\begin{tabular}{ | c   p{18cm} |}
		\hline
		\cellcolor{black}\rotcell{\large\textbf{\textcolor{white}{Elektrische Leiter}}}  &
		\setlength{\extrarowheight}{10pt}	
		
		\begin{tabular}{L{5cm} L{6.5cm} L{5.3cm}}
			&&\\
			
			\rowcolor[rgb]{1,1,1}
			\multicolumn{3}{l}{\parbox{17cm}{\begin{itemize}
						\item[(1)] Das elektrische Potential $\phi$ besitzt im Innern und auf der Oberfläche denselben Wert (Oberflächen sind Äquipotentialflächen).
						\item[(2)] Das E-Feld verschwindet im Innern und ist ausserhalb orthogonal auf die Oberfläche
						\[
						\vec{E} = \frac{\sigma}{\varepsilon_0}\hat{n}
						\]
						\begin{flushright}
							($\sigma$ lokale Flächenladungsdichte,\\
							$\vec{n}$ Normalenvektor)
						\end{flushright}
						\item[(3)] Je kleiner Krümmung der Oberfläche, umso grösser die Oberflächeladungsdichte $\sigma$.
						\item[(4)] Die Gesamtladung ist durch Integration über die Oberfläche gegeben
						\[
						q = \int_A \sigma \diff a = \varepsilon_0\int_A E\diff a
						\]
			\end{itemize}}}
			\\[70pt]

			
		
		\end{tabular}\\[10pt]
		\hline
	\end{tabular}\\

	
	
	
	
	
